\chapter*{Introduction}
The handshake event, commonly used, is a natural human interaction and is extensively used worldwide in events like: greetings, introduction routine between human beings and agreements. 
The scientific approach to handshake between a human and a robot, therefore, must intrinsically deal with low levels human interactions and some assumption must be done in order to focus on the task.
The handshake event can be divided in two parts: the approaching and the handshaking. This work is focused on the haptic sense involved during the handshake, knowing that the approaching is mainly executed using non haptic senses(f.i. vision).\\ %The full routine of the handshake event can be summarized in a starting time, a duration. Assuming the human to be in a leading position both, the starting point and the duration of the handshake are decided by the human.
A parameter involved in the handshake is the consensus; it allow the human to evaluate a handshake mixing aspects like: duration of the event, dynamics, force exchanged.
An important part of this work is to test different controllers with the purpose of evaluating the consensus using closed loop controllers. 
  
%Many research teams all over the world are focused on the human-robot physical interaction, this opens the topic to an interesting scientific discussion.  
