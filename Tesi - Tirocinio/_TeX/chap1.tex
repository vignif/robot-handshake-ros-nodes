\chapter*{Introduction}
The following thesis wants to explore the phases of the handshake between a human and a robot, reaching a consensus in the event and giving hints for future development in the interaction human-machine. The consensus can be considered as a parameter that allow the human to evaluate a handshake mixing aspects like: duration of the event, dynamics, force exchanged etc.. 
The handshake is the most common human-human interaction and is extensively used worldwide in events like: greetings, introduction routine between human beings and agreements. 

\chapter{The state of the Art}
\section{}
\section{}
\section{}
\chapter{The Idea}
\chapter{Hardware setup}
\section{The Pisa/IIT qbhand}
\section{The Sensors}
\section{}
\section{Ros}


\chapter*{Conclusion}
This project applies learning techniques to MNIST handwritten dataset. As we can see in the previous confusion matrix the accuracy of the final work is $97.6\%$. The overall idea is to train \emph{autoenc1},  \emph{autoenc2} and \emph{softmax1} once per time and to crop the nets in order to have coherents dimension between network interconnections. At the end of this process we stack all the partial neural network together and the deep neural network come to life. \\The satisfaction behind this project can be experimented by running the file "MNIST\textunderscore drawsim.m" which is a matlab function that allows the user to draw a digit and returns the correct digit value 97,6 times over 100.