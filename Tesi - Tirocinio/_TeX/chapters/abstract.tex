
% If you wish to include an abstract, uncomment the lines below
%Abstract text
%\abstract{

\begin{abstract}
%\textbf{Closed loop approach to Human-Robot Handshake}

\vspace{60px}
%----------------------------------------------------------------------------------------
%	SECTION 1
%----------------------------------------------------------------------------------------
%%\section{Environment}
The following work is focusing on the Human-Robot hand interaction, specifically in the grasping force of the handshake. The handshake event between human beings is a well known task, it can to enable a communication between participants as a mixture of physical features like: grip force of the hand, velocity approach, duration of the handshake, oscillation frequency and amplitude of the arm. The hypothesis is is that in human hand-shaking force control there is a balance between an intrinsic (open–loop) and extrinsic (closed–loop) contribution. The target of this work is to set up an environment in order to test the hypothesis for the human-robot handshake grip force.
The environment is built using Robotic Operating System (ROS), which is managing messages among: a Pisa/IIT SoftHand and three FSR 400 sensors managed by an Arduino Uno.
A detailed force analysis is needed in order to evaluate the grasping force in the human-robot handshake. The method is presented in \ref{subsec:calibFSR} is providing an estimation for the force the human applies on the robot $F_{h}$ and in \ref{ch:openloop} is presented a method for estimating the force the robot excerts on the human $F_{r}$.
The hypothesis is to model the human response to the robot grasp as a dynamical system.
A sensorimotor delay is mimicking the reaction time of Central Nervous System (CNS), and varying the contribution of intrinsic and extrinsic force control modify the perceived handshake quality in the user study. 
%
%%\\ 
% Abstract text
 \end{abstract}


